\documentclass[]{article}
\usepackage{lmodern}
\usepackage{amssymb,amsmath}
\usepackage{ifxetex,ifluatex}
\usepackage{fixltx2e} % provides \textsubscript
\ifnum 0\ifxetex 1\fi\ifluatex 1\fi=0 % if pdftex
  \usepackage[T1]{fontenc}
  \usepackage[utf8]{inputenc}
\else % if luatex or xelatex
  \ifxetex
    \usepackage{mathspec}
  \else
    \usepackage{fontspec}
  \fi
  \defaultfontfeatures{Ligatures=TeX,Scale=MatchLowercase}
\fi
% use upquote if available, for straight quotes in verbatim environments
\IfFileExists{upquote.sty}{\usepackage{upquote}}{}
% use microtype if available
\IfFileExists{microtype.sty}{%
\usepackage{microtype}
\UseMicrotypeSet[protrusion]{basicmath} % disable protrusion for tt fonts
}{}
\usepackage[margin=1in]{geometry}
\usepackage{hyperref}
\hypersetup{unicode=true,
            pdftitle={Pendahuluan},
            pdfauthor={Muhammad Aswan Syahputra},
            pdfborder={0 0 0},
            breaklinks=true}
\urlstyle{same}  % don't use monospace font for urls
\usepackage{color}
\usepackage{fancyvrb}
\newcommand{\VerbBar}{|}
\newcommand{\VERB}{\Verb[commandchars=\\\{\}]}
\DefineVerbatimEnvironment{Highlighting}{Verbatim}{commandchars=\\\{\}}
% Add ',fontsize=\small' for more characters per line
\usepackage{framed}
\definecolor{shadecolor}{RGB}{248,248,248}
\newenvironment{Shaded}{\begin{snugshade}}{\end{snugshade}}
\newcommand{\AlertTok}[1]{\textcolor[rgb]{0.94,0.16,0.16}{#1}}
\newcommand{\AnnotationTok}[1]{\textcolor[rgb]{0.56,0.35,0.01}{\textbf{\textit{#1}}}}
\newcommand{\AttributeTok}[1]{\textcolor[rgb]{0.77,0.63,0.00}{#1}}
\newcommand{\BaseNTok}[1]{\textcolor[rgb]{0.00,0.00,0.81}{#1}}
\newcommand{\BuiltInTok}[1]{#1}
\newcommand{\CharTok}[1]{\textcolor[rgb]{0.31,0.60,0.02}{#1}}
\newcommand{\CommentTok}[1]{\textcolor[rgb]{0.56,0.35,0.01}{\textit{#1}}}
\newcommand{\CommentVarTok}[1]{\textcolor[rgb]{0.56,0.35,0.01}{\textbf{\textit{#1}}}}
\newcommand{\ConstantTok}[1]{\textcolor[rgb]{0.00,0.00,0.00}{#1}}
\newcommand{\ControlFlowTok}[1]{\textcolor[rgb]{0.13,0.29,0.53}{\textbf{#1}}}
\newcommand{\DataTypeTok}[1]{\textcolor[rgb]{0.13,0.29,0.53}{#1}}
\newcommand{\DecValTok}[1]{\textcolor[rgb]{0.00,0.00,0.81}{#1}}
\newcommand{\DocumentationTok}[1]{\textcolor[rgb]{0.56,0.35,0.01}{\textbf{\textit{#1}}}}
\newcommand{\ErrorTok}[1]{\textcolor[rgb]{0.64,0.00,0.00}{\textbf{#1}}}
\newcommand{\ExtensionTok}[1]{#1}
\newcommand{\FloatTok}[1]{\textcolor[rgb]{0.00,0.00,0.81}{#1}}
\newcommand{\FunctionTok}[1]{\textcolor[rgb]{0.00,0.00,0.00}{#1}}
\newcommand{\ImportTok}[1]{#1}
\newcommand{\InformationTok}[1]{\textcolor[rgb]{0.56,0.35,0.01}{\textbf{\textit{#1}}}}
\newcommand{\KeywordTok}[1]{\textcolor[rgb]{0.13,0.29,0.53}{\textbf{#1}}}
\newcommand{\NormalTok}[1]{#1}
\newcommand{\OperatorTok}[1]{\textcolor[rgb]{0.81,0.36,0.00}{\textbf{#1}}}
\newcommand{\OtherTok}[1]{\textcolor[rgb]{0.56,0.35,0.01}{#1}}
\newcommand{\PreprocessorTok}[1]{\textcolor[rgb]{0.56,0.35,0.01}{\textit{#1}}}
\newcommand{\RegionMarkerTok}[1]{#1}
\newcommand{\SpecialCharTok}[1]{\textcolor[rgb]{0.00,0.00,0.00}{#1}}
\newcommand{\SpecialStringTok}[1]{\textcolor[rgb]{0.31,0.60,0.02}{#1}}
\newcommand{\StringTok}[1]{\textcolor[rgb]{0.31,0.60,0.02}{#1}}
\newcommand{\VariableTok}[1]{\textcolor[rgb]{0.00,0.00,0.00}{#1}}
\newcommand{\VerbatimStringTok}[1]{\textcolor[rgb]{0.31,0.60,0.02}{#1}}
\newcommand{\WarningTok}[1]{\textcolor[rgb]{0.56,0.35,0.01}{\textbf{\textit{#1}}}}
\usepackage{graphicx,grffile}
\makeatletter
\def\maxwidth{\ifdim\Gin@nat@width>\linewidth\linewidth\else\Gin@nat@width\fi}
\def\maxheight{\ifdim\Gin@nat@height>\textheight\textheight\else\Gin@nat@height\fi}
\makeatother
% Scale images if necessary, so that they will not overflow the page
% margins by default, and it is still possible to overwrite the defaults
% using explicit options in \includegraphics[width, height, ...]{}
\setkeys{Gin}{width=\maxwidth,height=\maxheight,keepaspectratio}
\IfFileExists{parskip.sty}{%
\usepackage{parskip}
}{% else
\setlength{\parindent}{0pt}
\setlength{\parskip}{6pt plus 2pt minus 1pt}
}
\setlength{\emergencystretch}{3em}  % prevent overfull lines
\providecommand{\tightlist}{%
  \setlength{\itemsep}{0pt}\setlength{\parskip}{0pt}}
\setcounter{secnumdepth}{0}
% Redefines (sub)paragraphs to behave more like sections
\ifx\paragraph\undefined\else
\let\oldparagraph\paragraph
\renewcommand{\paragraph}[1]{\oldparagraph{#1}\mbox{}}
\fi
\ifx\subparagraph\undefined\else
\let\oldsubparagraph\subparagraph
\renewcommand{\subparagraph}[1]{\oldsubparagraph{#1}\mbox{}}
\fi

%%% Use protect on footnotes to avoid problems with footnotes in titles
\let\rmarkdownfootnote\footnote%
\def\footnote{\protect\rmarkdownfootnote}

%%% Change title format to be more compact
\usepackage{titling}

% Create subtitle command for use in maketitle
\providecommand{\subtitle}[1]{
  \posttitle{
    \begin{center}\large#1\end{center}
    }
}

\setlength{\droptitle}{-2em}

  \title{Pendahuluan}
    \pretitle{\vspace{\droptitle}\centering\huge}
  \posttitle{\par}
    \author{Muhammad Aswan Syahputra}
    \preauthor{\centering\large\emph}
  \postauthor{\par}
      \predate{\centering\large\emph}
  \postdate{\par}
    \date{4/9/2019}


\begin{document}
\maketitle

{
\setcounter{tocdepth}{2}
\tableofcontents
}
\hypertarget{r-markdown}{%
\subsection{R Markdown}\label{r-markdown}}

Ini merupakan dokumen R Markdown yang dapat digunakan untuk membuat
dokumen HTML, PDF, dan bahkan dokumen berekstensi docx atau odt. Anda
dapat membuat dokumen tulisan, salindia presentasi, dan laman web statis
maupun interaktif dengan melalui R Markdown. Penggunaan R Markdown dalam
proyek analisis data akan membuat alur kerja menjadi lebih mudah dan
\emph{reproducible}. Informasi lebih lanjut mengenai R Markdown dapat
dilihat pada pranala \href{http://rmarkdown.rstudio.com}{ini}.

Kode R dapat dimasukan ke dalam dokumen R Markdown dengan menggunakan
\emph{chunck} yang dimulai dengan penanda tiga \emph{backtick} ````'
(dibuat dengan klik tombol Insert - R). Contoh cara untuk penulisan kode
R kedalam dokumen R Markdown adalah sebagai berikut:

\begin{Shaded}
\begin{Highlighting}[]
\KeywordTok{head}\NormalTok{(mtcars) }\CommentTok{# melihat 6 baris pertama dari data mtcars, mtcars adalah data bawaan yang tersedia di R}
\end{Highlighting}
\end{Shaded}

\begin{verbatim}
##                    mpg cyl disp  hp drat    wt  qsec vs am gear carb
## Mazda RX4         21.0   6  160 110 3.90 2.620 16.46  0  1    4    4
## Mazda RX4 Wag     21.0   6  160 110 3.90 2.875 17.02  0  1    4    4
## Datsun 710        22.8   4  108  93 3.85 2.320 18.61  1  1    4    1
## Hornet 4 Drive    21.4   6  258 110 3.08 3.215 19.44  1  0    3    1
## Hornet Sportabout 18.7   8  360 175 3.15 3.440 17.02  0  0    3    2
## Valiant           18.1   6  225 105 2.76 3.460 20.22  1  0    3    1
\end{verbatim}

Jika ingin menjalankan kode R dalam \emph{chunck} tersebut, Anda dapat
menggunakan pemintas Ctrl + Enter (menjalankan satu baris kode) atau
Ctrl + Shift + Enter (menjalankan semua kode dalam \emph{chunck}).
Sekarang buatlah \emph{chunck} baru yang isinya adalah baris kode R
berikut: (Petunjuk: Gunakan Ctrl + Alt + I)

\begin{Shaded}
\begin{Highlighting}[]
\KeywordTok{filled.contour}\NormalTok{(volcano,}
               \DataTypeTok{color.palette =}\NormalTok{ terrain.colors, }
               \DataTypeTok{plot.title =} \KeywordTok{title}\NormalTok{(}\StringTok{"Topografi Gunung Maunga Whau"}\NormalTok{), }
               \DataTypeTok{key.title =} \KeywordTok{title}\NormalTok{(}\StringTok{"Tinggi}\CharTok{\textbackslash{}n}\StringTok{(meter)"}\NormalTok{))}
\end{Highlighting}
\end{Shaded}

\includegraphics{001_pendahuluan_files/figure-latex/unnamed-chunk-2-1.pdf}

Setelah selesai membuat dokumen R Markdown yang berisikan konten tulisan
beserta kode R, Anda dapat klik tombol \textbf{Knit} untuk menghasilkan
dokumen baru sesuai dengan format dokumen yang diinginkan. Dalam contoh
ini format dokumen keluaran R Markdown setelah menjalankan \textbf{Knit}
adalah dokumen HTML. Anda dapat mengatur format dokumen keluaran dengan
cara mengubah baris \emph{output} pada YAML metadata (lihat baris paling
atas dokumen ini) seperti contoh berikut:

\begin{verbatim}
---
title: "Pendahuluan"
author: "Muhammad Aswan Syahputra"
date: "4/9/2019"
output: pdf_document
editor_options: 
  chunk_output_type: console
---
\end{verbatim}

\hypertarget{struktur-data}{%
\subsection{Struktur Data}\label{struktur-data}}

Struktur data pada R dapat dikategorikan berdasarkan dimensi dan
homogenitas dari elemen. Data satu dimensi dengan elemen yang homogen
disebut sebagai \emph{atomic vector}, sedangkan jika heterogen disebut
sebagai \emph{list}. Cara untuk membuat \emph{atomic vectors} adalah
dengan menggunakan fungsi \texttt{c()}, sedangkan untuk \emph{list}
dapat dibuat dengan menggunakan fungsi \texttt{list()}. Salah satu cara
untuk memberikan nama pada objek data adalah dengan menggunakan tanda
\texttt{\textless{}-}. Perhatikan contoh berikut:

\begin{Shaded}
\begin{Highlighting}[]
\KeywordTok{c}\NormalTok{(}\DecValTok{1}\NormalTok{, }\DecValTok{2}\NormalTok{, }\DecValTok{3}\NormalTok{, }\DecValTok{4}\NormalTok{)}
\end{Highlighting}
\end{Shaded}

\begin{verbatim}
## [1] 1 2 3 4
\end{verbatim}

\begin{Shaded}
\begin{Highlighting}[]
\KeywordTok{c}\NormalTok{(}\StringTok{"r"}\NormalTok{, }\StringTok{"academy"}\NormalTok{, }\StringTok{"telkom"}\NormalTok{, }\StringTok{"university"}\NormalTok{)}
\end{Highlighting}
\end{Shaded}

\begin{verbatim}
## [1] "r"          "academy"    "telkom"     "university"
\end{verbatim}

\begin{Shaded}
\begin{Highlighting}[]
\KeywordTok{list}\NormalTok{(}\DecValTok{15}\NormalTok{, }\StringTok{"r"}\NormalTok{, }\StringTok{"TRUE"}\NormalTok{, 24L)}
\end{Highlighting}
\end{Shaded}

\begin{verbatim}
## [[1]]
## [1] 15
## 
## [[2]]
## [1] "r"
## 
## [[3]]
## [1] "TRUE"
## 
## [[4]]
## [1] 24
\end{verbatim}

\begin{Shaded}
\begin{Highlighting}[]
\NormalTok{huruf_vokal <-}\StringTok{ }\KeywordTok{c}\NormalTok{(}\StringTok{"a"}\NormalTok{, }\StringTok{"i"}\NormalTok{, }\StringTok{"u"}\NormalTok{, }\StringTok{"e"}\NormalTok{, }\StringTok{"o"}\NormalTok{) }\CommentTok{# objek data tersimpan dengan nama 'huruf_vokal', namun tidak tercetak pada konsol}
\NormalTok{huruf_vokal }\CommentTok{# mencetak objek data dengan nama 'huruf_vokal' pada konsol}
\end{Highlighting}
\end{Shaded}

\begin{verbatim}
## [1] "a" "i" "u" "e" "o"
\end{verbatim}

Jenis dari data dapat diketahui dengan menggunakan fungsi
\texttt{typeof()}. Dapatkah Anda mengetahui jenis data dari
`huruf\_vokal' diatas? Bagaimana jika Anda membuat objek data dengan
menggunakan fungsi \texttt{c()} namun jenis elemennya berbeda-beda?
Dapatkah Anda menjelaskannya? Isilah '\_\_\_' dengan jawaban yang
sesuai!

\begin{Shaded}
\begin{Highlighting}[]
\NormalTok{huruf_vokal<-}\KeywordTok{c}\NormalTok{(}\StringTok{"a"}\NormalTok{, }\StringTok{"i"}\NormalTok{, }\StringTok{"u"}\NormalTok{, }\StringTok{"e"}\NormalTok{, }\StringTok{"o"}\NormalTok{)}

\NormalTok{beragam <-}\StringTok{ }\KeywordTok{c}\NormalTok{(}\FloatTok{2.7}\NormalTok{, }\StringTok{"berbeda"}\NormalTok{, }\OtherTok{TRUE}\NormalTok{, 4L) }\CommentTok{# 2.7 bertipe double, "berbeda" bertipe character, TRUE bertipe logical, 4L bertipe integer}
\KeywordTok{typeof}\NormalTok{(beragam) }\CommentTok{# cek tipe dari objek data dengan nama 'beragam'}
\end{Highlighting}
\end{Shaded}

\begin{verbatim}
## [1] "character"
\end{verbatim}

Struktur data dua dimensi merupakan yang paling banyak digunakan di R,
yaitu matrix dan dataframe yang dapat dibuat dengan menggunakan fungsi
\texttt{matrix()} dan \texttt{data.frame()}. Kedua data tersebut umumnya
jarang dibuat secara langsung di R, notabene berasal dari berkas luar
atau merupakan hasil dari penggunaan fungsi. Prinsipnya suatu dataframe
merupakan gabungan dari beberapa data satu dimensi dengan jumlah yang
sama, umumnya adalah \emph{atomic vectors}. Menurut Anda, dapatkah suatu
frame tersusun atas beberapa \emph{list} dengan jumlah yang sama?

\hypertarget{fungsi}{%
\subsection{Fungsi}\label{fungsi}}

Fungsi memiliki tugas utama untuk mengolah suatu \emph{input} menjadi
\emph{output}. Anda dapat melihat dan membaca dokumentasi dari suatu
fungsi dengan menjalankan \texttt{?nama\_fungsi} atau
\texttt{help(nama\_fungsi)}. Di bawah ini merupakan beberapa fungsi
dasar yang dapat Anda gunakan untuk mengolah objek data dengan nama
`iris' sebagai \emph{input}. Anda dipersilakan untuk menekan tombol
\textbf{Knit} (Ctrl + Shift + K) untuk menghasilkan dokumen html. Dari
dokumen html tersebut, dapatkah Anda menebak mana fungsi yang berjenis
`\emph{changing the environment}' dan mana yang berjenis
`\emph{calculating value}'?

\begin{Shaded}
\begin{Highlighting}[]
\NormalTok{iris }\CommentTok{# mencetak data di konsol}
\end{Highlighting}
\end{Shaded}

\begin{verbatim}
##     Sepal.Length Sepal.Width Petal.Length Petal.Width    Species
## 1            5.1         3.5          1.4         0.2     setosa
## 2            4.9         3.0          1.4         0.2     setosa
## 3            4.7         3.2          1.3         0.2     setosa
## 4            4.6         3.1          1.5         0.2     setosa
## 5            5.0         3.6          1.4         0.2     setosa
## 6            5.4         3.9          1.7         0.4     setosa
## 7            4.6         3.4          1.4         0.3     setosa
## 8            5.0         3.4          1.5         0.2     setosa
## 9            4.4         2.9          1.4         0.2     setosa
## 10           4.9         3.1          1.5         0.1     setosa
## 11           5.4         3.7          1.5         0.2     setosa
## 12           4.8         3.4          1.6         0.2     setosa
## 13           4.8         3.0          1.4         0.1     setosa
## 14           4.3         3.0          1.1         0.1     setosa
## 15           5.8         4.0          1.2         0.2     setosa
## 16           5.7         4.4          1.5         0.4     setosa
## 17           5.4         3.9          1.3         0.4     setosa
## 18           5.1         3.5          1.4         0.3     setosa
## 19           5.7         3.8          1.7         0.3     setosa
## 20           5.1         3.8          1.5         0.3     setosa
## 21           5.4         3.4          1.7         0.2     setosa
## 22           5.1         3.7          1.5         0.4     setosa
## 23           4.6         3.6          1.0         0.2     setosa
## 24           5.1         3.3          1.7         0.5     setosa
## 25           4.8         3.4          1.9         0.2     setosa
## 26           5.0         3.0          1.6         0.2     setosa
## 27           5.0         3.4          1.6         0.4     setosa
## 28           5.2         3.5          1.5         0.2     setosa
## 29           5.2         3.4          1.4         0.2     setosa
## 30           4.7         3.2          1.6         0.2     setosa
## 31           4.8         3.1          1.6         0.2     setosa
## 32           5.4         3.4          1.5         0.4     setosa
## 33           5.2         4.1          1.5         0.1     setosa
## 34           5.5         4.2          1.4         0.2     setosa
## 35           4.9         3.1          1.5         0.2     setosa
## 36           5.0         3.2          1.2         0.2     setosa
## 37           5.5         3.5          1.3         0.2     setosa
## 38           4.9         3.6          1.4         0.1     setosa
## 39           4.4         3.0          1.3         0.2     setosa
## 40           5.1         3.4          1.5         0.2     setosa
## 41           5.0         3.5          1.3         0.3     setosa
## 42           4.5         2.3          1.3         0.3     setosa
## 43           4.4         3.2          1.3         0.2     setosa
## 44           5.0         3.5          1.6         0.6     setosa
## 45           5.1         3.8          1.9         0.4     setosa
## 46           4.8         3.0          1.4         0.3     setosa
## 47           5.1         3.8          1.6         0.2     setosa
## 48           4.6         3.2          1.4         0.2     setosa
## 49           5.3         3.7          1.5         0.2     setosa
## 50           5.0         3.3          1.4         0.2     setosa
## 51           7.0         3.2          4.7         1.4 versicolor
## 52           6.4         3.2          4.5         1.5 versicolor
## 53           6.9         3.1          4.9         1.5 versicolor
## 54           5.5         2.3          4.0         1.3 versicolor
## 55           6.5         2.8          4.6         1.5 versicolor
## 56           5.7         2.8          4.5         1.3 versicolor
## 57           6.3         3.3          4.7         1.6 versicolor
## 58           4.9         2.4          3.3         1.0 versicolor
## 59           6.6         2.9          4.6         1.3 versicolor
## 60           5.2         2.7          3.9         1.4 versicolor
## 61           5.0         2.0          3.5         1.0 versicolor
## 62           5.9         3.0          4.2         1.5 versicolor
## 63           6.0         2.2          4.0         1.0 versicolor
## 64           6.1         2.9          4.7         1.4 versicolor
## 65           5.6         2.9          3.6         1.3 versicolor
## 66           6.7         3.1          4.4         1.4 versicolor
## 67           5.6         3.0          4.5         1.5 versicolor
## 68           5.8         2.7          4.1         1.0 versicolor
## 69           6.2         2.2          4.5         1.5 versicolor
## 70           5.6         2.5          3.9         1.1 versicolor
## 71           5.9         3.2          4.8         1.8 versicolor
## 72           6.1         2.8          4.0         1.3 versicolor
## 73           6.3         2.5          4.9         1.5 versicolor
## 74           6.1         2.8          4.7         1.2 versicolor
## 75           6.4         2.9          4.3         1.3 versicolor
## 76           6.6         3.0          4.4         1.4 versicolor
## 77           6.8         2.8          4.8         1.4 versicolor
## 78           6.7         3.0          5.0         1.7 versicolor
## 79           6.0         2.9          4.5         1.5 versicolor
## 80           5.7         2.6          3.5         1.0 versicolor
## 81           5.5         2.4          3.8         1.1 versicolor
## 82           5.5         2.4          3.7         1.0 versicolor
## 83           5.8         2.7          3.9         1.2 versicolor
## 84           6.0         2.7          5.1         1.6 versicolor
## 85           5.4         3.0          4.5         1.5 versicolor
## 86           6.0         3.4          4.5         1.6 versicolor
## 87           6.7         3.1          4.7         1.5 versicolor
## 88           6.3         2.3          4.4         1.3 versicolor
## 89           5.6         3.0          4.1         1.3 versicolor
## 90           5.5         2.5          4.0         1.3 versicolor
## 91           5.5         2.6          4.4         1.2 versicolor
## 92           6.1         3.0          4.6         1.4 versicolor
## 93           5.8         2.6          4.0         1.2 versicolor
## 94           5.0         2.3          3.3         1.0 versicolor
## 95           5.6         2.7          4.2         1.3 versicolor
## 96           5.7         3.0          4.2         1.2 versicolor
## 97           5.7         2.9          4.2         1.3 versicolor
## 98           6.2         2.9          4.3         1.3 versicolor
## 99           5.1         2.5          3.0         1.1 versicolor
## 100          5.7         2.8          4.1         1.3 versicolor
## 101          6.3         3.3          6.0         2.5  virginica
## 102          5.8         2.7          5.1         1.9  virginica
## 103          7.1         3.0          5.9         2.1  virginica
## 104          6.3         2.9          5.6         1.8  virginica
## 105          6.5         3.0          5.8         2.2  virginica
## 106          7.6         3.0          6.6         2.1  virginica
## 107          4.9         2.5          4.5         1.7  virginica
## 108          7.3         2.9          6.3         1.8  virginica
## 109          6.7         2.5          5.8         1.8  virginica
## 110          7.2         3.6          6.1         2.5  virginica
## 111          6.5         3.2          5.1         2.0  virginica
## 112          6.4         2.7          5.3         1.9  virginica
## 113          6.8         3.0          5.5         2.1  virginica
## 114          5.7         2.5          5.0         2.0  virginica
## 115          5.8         2.8          5.1         2.4  virginica
## 116          6.4         3.2          5.3         2.3  virginica
## 117          6.5         3.0          5.5         1.8  virginica
## 118          7.7         3.8          6.7         2.2  virginica
## 119          7.7         2.6          6.9         2.3  virginica
## 120          6.0         2.2          5.0         1.5  virginica
## 121          6.9         3.2          5.7         2.3  virginica
## 122          5.6         2.8          4.9         2.0  virginica
## 123          7.7         2.8          6.7         2.0  virginica
## 124          6.3         2.7          4.9         1.8  virginica
## 125          6.7         3.3          5.7         2.1  virginica
## 126          7.2         3.2          6.0         1.8  virginica
## 127          6.2         2.8          4.8         1.8  virginica
## 128          6.1         3.0          4.9         1.8  virginica
## 129          6.4         2.8          5.6         2.1  virginica
## 130          7.2         3.0          5.8         1.6  virginica
## 131          7.4         2.8          6.1         1.9  virginica
## 132          7.9         3.8          6.4         2.0  virginica
## 133          6.4         2.8          5.6         2.2  virginica
## 134          6.3         2.8          5.1         1.5  virginica
## 135          6.1         2.6          5.6         1.4  virginica
## 136          7.7         3.0          6.1         2.3  virginica
## 137          6.3         3.4          5.6         2.4  virginica
## 138          6.4         3.1          5.5         1.8  virginica
## 139          6.0         3.0          4.8         1.8  virginica
## 140          6.9         3.1          5.4         2.1  virginica
## 141          6.7         3.1          5.6         2.4  virginica
## 142          6.9         3.1          5.1         2.3  virginica
## 143          5.8         2.7          5.1         1.9  virginica
## 144          6.8         3.2          5.9         2.3  virginica
## 145          6.7         3.3          5.7         2.5  virginica
## 146          6.7         3.0          5.2         2.3  virginica
## 147          6.3         2.5          5.0         1.9  virginica
## 148          6.5         3.0          5.2         2.0  virginica
## 149          6.2         3.4          5.4         2.3  virginica
## 150          5.9         3.0          5.1         1.8  virginica
\end{verbatim}

\begin{Shaded}
\begin{Highlighting}[]
\KeywordTok{dim}\NormalTok{(iris) }\CommentTok{# memberikan dimensi dari data. Angka pertama merupakan jumlah baris dan angka kedua adalah jumlah kolom}
\end{Highlighting}
\end{Shaded}

\begin{verbatim}
## [1] 150   5
\end{verbatim}

\begin{Shaded}
\begin{Highlighting}[]
\KeywordTok{str}\NormalTok{(iris) }\CommentTok{# mencetak struktur dari data (jumlah baris, jumlah kolom, nama kolom, jenis data pada kolom)}
\end{Highlighting}
\end{Shaded}

\begin{verbatim}
## 'data.frame':    150 obs. of  5 variables:
##  $ Sepal.Length: num  5.1 4.9 4.7 4.6 5 5.4 4.6 5 4.4 4.9 ...
##  $ Sepal.Width : num  3.5 3 3.2 3.1 3.6 3.9 3.4 3.4 2.9 3.1 ...
##  $ Petal.Length: num  1.4 1.4 1.3 1.5 1.4 1.7 1.4 1.5 1.4 1.5 ...
##  $ Petal.Width : num  0.2 0.2 0.2 0.2 0.2 0.4 0.3 0.2 0.2 0.1 ...
##  $ Species     : Factor w/ 3 levels "setosa","versicolor",..: 1 1 1 1 1 1 1 1 1 1 ...
\end{verbatim}

\begin{Shaded}
\begin{Highlighting}[]
\KeywordTok{colnames}\NormalTok{(iris) }\CommentTok{# mencetak nama dari setiap kolom}
\end{Highlighting}
\end{Shaded}

\begin{verbatim}
## [1] "Sepal.Length" "Sepal.Width"  "Petal.Length" "Petal.Width" 
## [5] "Species"
\end{verbatim}

\begin{Shaded}
\begin{Highlighting}[]
\KeywordTok{head}\NormalTok{(iris) }\CommentTok{# mencetak 6 baris pertama pada data}
\end{Highlighting}
\end{Shaded}

\begin{verbatim}
##   Sepal.Length Sepal.Width Petal.Length Petal.Width Species
## 1          5.1         3.5          1.4         0.2  setosa
## 2          4.9         3.0          1.4         0.2  setosa
## 3          4.7         3.2          1.3         0.2  setosa
## 4          4.6         3.1          1.5         0.2  setosa
## 5          5.0         3.6          1.4         0.2  setosa
## 6          5.4         3.9          1.7         0.4  setosa
\end{verbatim}

\begin{Shaded}
\begin{Highlighting}[]
\KeywordTok{head}\NormalTok{(iris, }\DecValTok{10}\NormalTok{) }\CommentTok{# mencetak 10 baris pertama pada data}
\end{Highlighting}
\end{Shaded}

\begin{verbatim}
##    Sepal.Length Sepal.Width Petal.Length Petal.Width Species
## 1           5.1         3.5          1.4         0.2  setosa
## 2           4.9         3.0          1.4         0.2  setosa
## 3           4.7         3.2          1.3         0.2  setosa
## 4           4.6         3.1          1.5         0.2  setosa
## 5           5.0         3.6          1.4         0.2  setosa
## 6           5.4         3.9          1.7         0.4  setosa
## 7           4.6         3.4          1.4         0.3  setosa
## 8           5.0         3.4          1.5         0.2  setosa
## 9           4.4         2.9          1.4         0.2  setosa
## 10          4.9         3.1          1.5         0.1  setosa
\end{verbatim}

\begin{Shaded}
\begin{Highlighting}[]
\KeywordTok{tail}\NormalTok{(iris) }\CommentTok{# mencetak 6 baris terakhir pada data}
\end{Highlighting}
\end{Shaded}

\begin{verbatim}
##     Sepal.Length Sepal.Width Petal.Length Petal.Width   Species
## 145          6.7         3.3          5.7         2.5 virginica
## 146          6.7         3.0          5.2         2.3 virginica
## 147          6.3         2.5          5.0         1.9 virginica
## 148          6.5         3.0          5.2         2.0 virginica
## 149          6.2         3.4          5.4         2.3 virginica
## 150          5.9         3.0          5.1         1.8 virginica
\end{verbatim}

\begin{Shaded}
\begin{Highlighting}[]
\KeywordTok{tail}\NormalTok{(iris, }\DecValTok{10}\NormalTok{) }\CommentTok{# mencetak 10 baris terakhir pada data}
\end{Highlighting}
\end{Shaded}

\begin{verbatim}
##     Sepal.Length Sepal.Width Petal.Length Petal.Width   Species
## 141          6.7         3.1          5.6         2.4 virginica
## 142          6.9         3.1          5.1         2.3 virginica
## 143          5.8         2.7          5.1         1.9 virginica
## 144          6.8         3.2          5.9         2.3 virginica
## 145          6.7         3.3          5.7         2.5 virginica
## 146          6.7         3.0          5.2         2.3 virginica
## 147          6.3         2.5          5.0         1.9 virginica
## 148          6.5         3.0          5.2         2.0 virginica
## 149          6.2         3.4          5.4         2.3 virginica
## 150          5.9         3.0          5.1         1.8 virginica
\end{verbatim}

\begin{Shaded}
\begin{Highlighting}[]
\KeywordTok{summary}\NormalTok{(iris) }\CommentTok{# mencetak rangkuman data}
\end{Highlighting}
\end{Shaded}

\begin{verbatim}
##   Sepal.Length    Sepal.Width     Petal.Length    Petal.Width   
##  Min.   :4.300   Min.   :2.000   Min.   :1.000   Min.   :0.100  
##  1st Qu.:5.100   1st Qu.:2.800   1st Qu.:1.600   1st Qu.:0.300  
##  Median :5.800   Median :3.000   Median :4.350   Median :1.300  
##  Mean   :5.843   Mean   :3.057   Mean   :3.758   Mean   :1.199  
##  3rd Qu.:6.400   3rd Qu.:3.300   3rd Qu.:5.100   3rd Qu.:1.800  
##  Max.   :7.900   Max.   :4.400   Max.   :6.900   Max.   :2.500  
##        Species  
##  setosa    :50  
##  versicolor:50  
##  virginica :50  
##                 
##                 
## 
\end{verbatim}

\begin{Shaded}
\begin{Highlighting}[]
\NormalTok{iris[}\DecValTok{1}\NormalTok{, ] }\CommentTok{# subset data pada baris 1}
\end{Highlighting}
\end{Shaded}

\begin{verbatim}
##   Sepal.Length Sepal.Width Petal.Length Petal.Width Species
## 1          5.1         3.5          1.4         0.2  setosa
\end{verbatim}

\begin{Shaded}
\begin{Highlighting}[]
\NormalTok{iris[}\KeywordTok{c}\NormalTok{(}\DecValTok{1}\NormalTok{, }\DecValTok{6}\NormalTok{, }\DecValTok{12}\NormalTok{),] }\CommentTok{# subset data pada baris 1, 6, dan 12}
\end{Highlighting}
\end{Shaded}

\begin{verbatim}
##    Sepal.Length Sepal.Width Petal.Length Petal.Width Species
## 1           5.1         3.5          1.4         0.2  setosa
## 6           5.4         3.9          1.7         0.4  setosa
## 12          4.8         3.4          1.6         0.2  setosa
\end{verbatim}

\begin{Shaded}
\begin{Highlighting}[]
\NormalTok{iris[ ,}\DecValTok{2}\NormalTok{] }\CommentTok{# subset atau ekstrak data pada kolom 2}
\end{Highlighting}
\end{Shaded}

\begin{verbatim}
##   [1] 3.5 3.0 3.2 3.1 3.6 3.9 3.4 3.4 2.9 3.1 3.7 3.4 3.0 3.0 4.0 4.4 3.9
##  [18] 3.5 3.8 3.8 3.4 3.7 3.6 3.3 3.4 3.0 3.4 3.5 3.4 3.2 3.1 3.4 4.1 4.2
##  [35] 3.1 3.2 3.5 3.6 3.0 3.4 3.5 2.3 3.2 3.5 3.8 3.0 3.8 3.2 3.7 3.3 3.2
##  [52] 3.2 3.1 2.3 2.8 2.8 3.3 2.4 2.9 2.7 2.0 3.0 2.2 2.9 2.9 3.1 3.0 2.7
##  [69] 2.2 2.5 3.2 2.8 2.5 2.8 2.9 3.0 2.8 3.0 2.9 2.6 2.4 2.4 2.7 2.7 3.0
##  [86] 3.4 3.1 2.3 3.0 2.5 2.6 3.0 2.6 2.3 2.7 3.0 2.9 2.9 2.5 2.8 3.3 2.7
## [103] 3.0 2.9 3.0 3.0 2.5 2.9 2.5 3.6 3.2 2.7 3.0 2.5 2.8 3.2 3.0 3.8 2.6
## [120] 2.2 3.2 2.8 2.8 2.7 3.3 3.2 2.8 3.0 2.8 3.0 2.8 3.8 2.8 2.8 2.6 3.0
## [137] 3.4 3.1 3.0 3.1 3.1 3.1 2.7 3.2 3.3 3.0 2.5 3.0 3.4 3.0
\end{verbatim}

\begin{Shaded}
\begin{Highlighting}[]
\NormalTok{iris[}\DecValTok{2}\NormalTok{,}\DecValTok{2}\NormalTok{] }\CommentTok{# ekstrak nilai pada baris 2 dan kolom 2}
\end{Highlighting}
\end{Shaded}

\begin{verbatim}
## [1] 3
\end{verbatim}

\begin{Shaded}
\begin{Highlighting}[]
\NormalTok{iris}\OperatorTok{$}\NormalTok{Sepal.Length }\CommentTok{# ekstrak data pada kolom 'Sepal.Length'}
\end{Highlighting}
\end{Shaded}

\begin{verbatim}
##   [1] 5.1 4.9 4.7 4.6 5.0 5.4 4.6 5.0 4.4 4.9 5.4 4.8 4.8 4.3 5.8 5.7 5.4
##  [18] 5.1 5.7 5.1 5.4 5.1 4.6 5.1 4.8 5.0 5.0 5.2 5.2 4.7 4.8 5.4 5.2 5.5
##  [35] 4.9 5.0 5.5 4.9 4.4 5.1 5.0 4.5 4.4 5.0 5.1 4.8 5.1 4.6 5.3 5.0 7.0
##  [52] 6.4 6.9 5.5 6.5 5.7 6.3 4.9 6.6 5.2 5.0 5.9 6.0 6.1 5.6 6.7 5.6 5.8
##  [69] 6.2 5.6 5.9 6.1 6.3 6.1 6.4 6.6 6.8 6.7 6.0 5.7 5.5 5.5 5.8 6.0 5.4
##  [86] 6.0 6.7 6.3 5.6 5.5 5.5 6.1 5.8 5.0 5.6 5.7 5.7 6.2 5.1 5.7 6.3 5.8
## [103] 7.1 6.3 6.5 7.6 4.9 7.3 6.7 7.2 6.5 6.4 6.8 5.7 5.8 6.4 6.5 7.7 7.7
## [120] 6.0 6.9 5.6 7.7 6.3 6.7 7.2 6.2 6.1 6.4 7.2 7.4 7.9 6.4 6.3 6.1 7.7
## [137] 6.3 6.4 6.0 6.9 6.7 6.9 5.8 6.8 6.7 6.7 6.3 6.5 6.2 5.9
\end{verbatim}

\begin{Shaded}
\begin{Highlighting}[]
\KeywordTok{class}\NormalTok{(iris}\OperatorTok{$}\NormalTok{Species) }\CommentTok{# mengetahui jenis data dari suatu vektor}
\end{Highlighting}
\end{Shaded}

\begin{verbatim}
## [1] "factor"
\end{verbatim}

\begin{Shaded}
\begin{Highlighting}[]
\KeywordTok{length}\NormalTok{(iris}\OperatorTok{$}\NormalTok{Petal.Width) }\CommentTok{# mengetahui jumlah data pada suatu vektor}
\end{Highlighting}
\end{Shaded}

\begin{verbatim}
## [1] 150
\end{verbatim}

\begin{Shaded}
\begin{Highlighting}[]
\KeywordTok{levels}\NormalTok{(iris}\OperatorTok{$}\NormalTok{Species) }\CommentTok{# mengetahui level dari kolom berjenis faktor}
\end{Highlighting}
\end{Shaded}

\begin{verbatim}
## [1] "setosa"     "versicolor" "virginica"
\end{verbatim}

\begin{Shaded}
\begin{Highlighting}[]
\KeywordTok{length}\NormalTok{(}\KeywordTok{levels}\NormalTok{(iris}\OperatorTok{$}\NormalTok{Species)) }\CommentTok{# mengetahui jumlah level dari kolom berjenis faktor}
\end{Highlighting}
\end{Shaded}

\begin{verbatim}
## [1] 3
\end{verbatim}

\begin{Shaded}
\begin{Highlighting}[]
\KeywordTok{unique}\NormalTok{(iris}\OperatorTok{$}\NormalTok{Species) }\CommentTok{# mengetahui nilai unik dari kolom berjenis karakter atau faktor}
\end{Highlighting}
\end{Shaded}

\begin{verbatim}
## [1] setosa     versicolor virginica 
## Levels: setosa versicolor virginica
\end{verbatim}

\begin{Shaded}
\begin{Highlighting}[]
\KeywordTok{length}\NormalTok{(}\KeywordTok{unique}\NormalTok{(iris}\OperatorTok{$}\NormalTok{Species)) }\CommentTok{# mengetahui jumlah nilai unik dari kolom berjenis karakter atau faktor}
\end{Highlighting}
\end{Shaded}

\begin{verbatim}
## [1] 3
\end{verbatim}

\begin{Shaded}
\begin{Highlighting}[]
\KeywordTok{mean}\NormalTok{(iris}\OperatorTok{$}\NormalTok{Sepal.Length) }\CommentTok{# menghitung rerata}
\end{Highlighting}
\end{Shaded}

\begin{verbatim}
## [1] 5.843333
\end{verbatim}

\begin{Shaded}
\begin{Highlighting}[]
\KeywordTok{sd}\NormalTok{(iris}\OperatorTok{$}\NormalTok{Sepal.Length) }\CommentTok{# menghitung standar deviasi}
\end{Highlighting}
\end{Shaded}

\begin{verbatim}
## [1] 0.8280661
\end{verbatim}

\begin{Shaded}
\begin{Highlighting}[]
\KeywordTok{median}\NormalTok{(iris}\OperatorTok{$}\NormalTok{Sepal.Width) }\CommentTok{# menghitung median}
\end{Highlighting}
\end{Shaded}

\begin{verbatim}
## [1] 3
\end{verbatim}

\begin{Shaded}
\begin{Highlighting}[]
\KeywordTok{sum}\NormalTok{(iris}\OperatorTok{$}\NormalTok{Petal.Length) }\CommentTok{# menghitung jumlah total}
\end{Highlighting}
\end{Shaded}

\begin{verbatim}
## [1] 563.7
\end{verbatim}

\begin{Shaded}
\begin{Highlighting}[]
\KeywordTok{plot}\NormalTok{(iris) }\CommentTok{# membuat plot}
\end{Highlighting}
\end{Shaded}

\includegraphics{001_pendahuluan_files/figure-latex/unnamed-chunk-5-1.pdf}

\begin{Shaded}
\begin{Highlighting}[]
\KeywordTok{cor}\NormalTok{(iris[, }\DecValTok{-5}\NormalTok{]) }\CommentTok{# membuat matriks korelasi pada semua kolom berjenis numerik}
\end{Highlighting}
\end{Shaded}

\begin{verbatim}
##              Sepal.Length Sepal.Width Petal.Length Petal.Width
## Sepal.Length    1.0000000  -0.1175698    0.8717538   0.8179411
## Sepal.Width    -0.1175698   1.0000000   -0.4284401  -0.3661259
## Petal.Length    0.8717538  -0.4284401    1.0000000   0.9628654
## Petal.Width     0.8179411  -0.3661259    0.9628654   1.0000000
\end{verbatim}

\begin{Shaded}
\begin{Highlighting}[]
\KeywordTok{write.csv}\NormalTok{(iris, }\DataTypeTok{file =} \StringTok{"iris.csv"}\NormalTok{, }\DataTypeTok{row.names =} \OtherTok{FALSE}\NormalTok{) }\CommentTok{# menyimpan iris menjadi berkas csv }
\end{Highlighting}
\end{Shaded}


\end{document}
